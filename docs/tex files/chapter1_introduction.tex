% --- ΚΕΦΑΛΑΙΟ 1: ΕΙΣΑΓΩΓΗ ---
\section{Εισαγωγή}
\label{sec:eisagogi}

\subsection{Σκοπός και Στόχοι του Λογισμικού}
\label{sec:skopos_stoxoi}
Ο κύριος σκοπός της εφαρμογής (\eng{ScriptWise}) είναι η δημιουργία ενός σύγχρονου και διαδραστικού εκπαιδευτικού περιβάλλοντος για την εκμάθηση της γλώσσας προγραμματισμού \eng{JavaScript}. Η εφαρμογή σχεδιάστηκε με επίκεντρο τον χρήστη, υιοθετώντας αρχές της ενεργητικής μάθησης (\eng{active learning}), όπου ο εκπαιδευόμενος συμμετέχει ενεργά μέσω πρακτικών ασκήσεων και άμεσης ανατροφοδότησης, καθώς και της εξατομικευμένης μάθησης (\eng{personalized learning}), όπου το περιεχόμενο και ο ρυθμός προσαρμόζονται στις ατομικές του ανάγκες. Στόχος είναι η ενίσχυση της κατανόησης, η καλλιέργεια του ενδιαφέροντος για τον προγραμματισμό και η αποτελεσματική ανάπτυξη δεξιοτήτων.

Οι βασικοί στόχοι που τέθηκαν κατά την ανάπτυξη του λογισμικού, αντικατοπτρίζοντας τις παραπάνω μαθησιακές φιλοσοφίες, είναι οι εξής:
\begin{itemize}[leftmargin=*, noitemsep]
    \item \textbf{Παροχή Δομημένου Εκπαιδευτικού Υλικού:} Προσφορά σαφών και κατανοητών ενοτήτων διδασκαλίας που καλύπτουν από τις θεμελιώδεις αρχές έως και πιο προχωρημένες έννοιες της \eng{JavaScript}, υποστηρίζοντας την οικοδόμηση της γνώσης (\eng{constructivism}) βήμα προς βήμα.
    \item \textbf{Ενεργητική Μάθηση και Πρακτική:} Ενσωμάτωση διαδραστικού επεξεργαστή κώδικα (\eng{code editor}) εντός των μαθημάτων, επιτρέποντας στους χρήστες να πειραματιστούν και να ολοκληρώσουν ασκήσεις κώδικα σε πραγματικό χρόνο, προωθώντας τη μάθηση μέσω της πράξης (\eng{learning by doing}).
    \item \textbf{Συνεχής Αυτοαξιολόγηση και Ανατροφοδότηση:} Δυνατότητα αξιολόγησης της γνώσης μέσω ποικίλων \eng{quiz} (ανά ενότητα, επαναληπτικά, αξιολόγησης αρχικού επιπέδου και τελικής επανάληψης) με διαφορετικούς τύπους ερωτήσεων, παρέχοντας άμεση ανατροφοδότηση για την ενίσχυση της κατανόησης.
    \item \textbf{Εξατομίκευση της Μαθησιακής Εμπειρίας:}
    \begin{itemize}[leftmargin=*, noitemsep]
        \item Προσαρμογή της μαθησιακής διαδρομής βάσει των επιδόσεων του χρήστη (π.χ., πρόταση παράκαμψης μαθημάτων μετά από επιτυχή αξιολόγηση αρχικού επιπέδου).
        \item Προσαρμογή της παρουσίασης του περιεχομένου ανάλογα με τις δηλωμένες προτιμήσεις του χρήστη (π.χ., έμφαση σε οπτικό ή κειμενικό υλικό).
        \item Πρόταση επιπλέον υλικού και στοχευμένης επανάληψης σε περιοχές όπου ο χρήστης αντιμετωπίζει δυσκολίες, υποστηρίζοντας την καθοδηγούμενη ανακάλυψη (\eng{scaffolding}).
        \item Δυνατότητα επιλογής συγκεκριμένων "μαθησιακών μονοπατιών" (\eng{learning paths}) που ταιριάζουν στους στόχους του χρήστη.
    \end{itemize}
    \item \textbf{Παρακολούθηση και Οπτικοποίηση της Προόδου:} Καταγραφή και εμφάνιση των στατιστικών προόδου του χρήστη (π.χ., ολοκληρωμένα μαθήματα, επιδόσεις στα \eng{quiz}) μέσω γραφημάτων για την παροχή ανατροφοδότησης και την ενίσχυση του κινήτρου, σύμφωνα με τις αρχές της ορατής μάθησης (\eng{visible learning}).
\end{itemize}

\subsection{Βασικές Λειτουργίες της Εφαρμογής}
\label{sec:leitourgies_oria}
Η εφαρμογή αυτή δημιουργήθηκε για να κάνει την εκμάθηση της γλώσσας προγραμματισμού \eng{JavaScript} πιο εύκολη και προσαρμοσμένη στις ανάγκες του καθενός. Συγκεκριμένα, η πλατφόρμα μας προσφέρει:
\begin{itemize}[leftmargin=*, noitemsep]
    \item \textbf{Μαθήματα για όλα τα Επίπεδα:} Βρίσκεις υλικό και παραδείγματα κώδικα για να μάθεις \eng{JavaScript}, από τα πρώτα σου βήματα μέχρι πιο σύνθετα θέματα. Όλα είναι οργανωμένα σε σειρές μαθημάτων (\eng{courses}) και μικρότερα κεφάλαια (\eng{lessons}).
    \item \textbf{Γράψε Κώδικα Άμεσα:} Μέσα σε κάθε μάθημα, υπάρχει ένας επεξεργαστής κώδικα. Εκεί μπορείς να γράψεις τη δική σου \eng{JavaScript}, να τη δεις να "τρέχει" απευθείας στον \eng{browser} σου και να κάνεις τις ασκήσεις.
    \item \textbf{Τεστ για να Δεις τι Έμαθες:} Υπάρχουν διάφορα \eng{quiz} (πολλαπλής επιλογής, σωστό/λάθος, συμπλήρωσης κενού) για να τεστάρεις τις γνώσεις σου μετά από κάθε ενότητα, ή για να κάνεις μια γενική επανάληψη. Θα βρεις επίσης τεστ για να δεις το αρχικό σου επίπεδο ή για μια τελική αξιολόγηση.
    \item \textbf{Δες την Πρόοδό σου:} Η εφαρμογή κρατάει αρχείο για τα μαθήματα που ολοκληρώνεις και πώς τα πήγες στα \eng{quiz}.
    \item \textbf{Μάθηση στα Μέτρα σου:} Η εφαρμογή προσπαθεί να σε βοηθήσει προσωπικά:
    \begin{itemize}[leftmargin=*, noitemsep]
        \item Αν ξέρεις ήδη κάποια πράγματα, ένα αρχικό τεστ μπορεί να σου προτείνει να προσπεράσεις κάποια από τα πρώτα μαθήματα.
        \item Αν δυσκολευτείς σε κάποιο τελικό \eng{quiz}, θα σου προτείνει μαθήματα για επανάληψη και επιπλέον υλικό από άλλες πηγές.
        \item Μπορείς να πεις αν προτιμάς να μαθαίνεις διαβάζοντας ή βλέποντας (π.χ. βίντεο), και η εφαρμογή θα προσπαθεί να σου δείχνει πρώτα αυτό που σου ταιριάζει.
        \item Μπορείς να διαλέξεις ένα "μονοπάτι μάθησης" (\eng{learning path}) που σε οδηγεί βήμα-βήμα ανάλογα με τους στόχους σου (π.χ. "\eng{Frontend Developer}").
    \end{itemize}
    \item \textbf{Τα Στατιστικά σου με μια Ματιά:} Βλέπεις πόσες μέρες στη σειρά μαθαίνεις (\eng{learning streak}), πόσα \eng{quiz} έχεις κάνει, και πώς τα πας γενικά στα \eng{quiz} μέσα από απλά γραφήματα.
    \item \textbf{Ο Λογαριασμός σου:} Μπορείς να κάνεις εγγραφή, να συνδέεσαι, να αποσυνδέεσαι και να αλλάζεις τα στοιχεία του προφίλ σου (όνομα, \eng{email}, \eng{password}) και τις προτιμήσεις μάθησης.
\end{itemize}

\subsection{Αναμενόμενοι Χρήστες}
\label{sec:anamenomenoi_xristes}
Η παρούσα εκπαιδευτική εφαρμογή σχεδιάστηκε με σκοπό να εξυπηρετήσει ένα ευρύ φάσμα εκπαιδευομένων που επιθυμούν να μάθουν ή να βελτιώσουν τις γνώσεις τους στη γλώσσα προγραμματισμού \eng{JavaScript}. Οι κύριες ομάδες χρηστών στις οποίες απευθυνόμαστε είναι:
\begin{itemize}[leftmargin=*, noitemsep]
    \item \textbf{Αρχάριοι στον Προγραμματισμό:} Άτομα που δεν έχουν προηγούμενη εμπειρία στον προγραμματισμό και επιθυμούν να ξεκινήσουν το ταξίδι τους με τη \eng{JavaScript} ως πρώτη γλώσσα.
    \item \textbf{Φοιτητές και Σπουδαστές:} Εκπαιδευόμενοι σε τμήματα πληροφορικής, \eng{web development} ή συναφείς τομείς που διδάσκονται \eng{JavaScript} και αναζητούν ένα συμπληρωματικό, διαδραστικό εργαλείο.
    \item \textbf{Προγραμματιστές Άλλων Γλωσσών:} Προγραμματιστές με εμπειρία σε άλλες γλώσσες που θέλουν να προσθέσουν τη \eng{JavaScript} στις δεξιότητές τους.
    \item \textbf{Επαγγελματίες για \eng{Upskilling/Reskilling}:} Επαγγελματίες που επιθυμούν να αποκτήσουν γνώσεις \eng{JavaScript} για να ενισχύσουν το επαγγελματικό τους προφίλ.
    \item \textbf{Άτομα που Μαθαίνουν Αυτοδίδακτα:} Όσοι προτιμούν να μαθαίνουν με τον δικό τους ρυθμό.
\end{itemize}